\documentclass[11pt,letterpaper]{article}
\usepackage{amsmath,amssymb}
\usepackage{graphicx}
\usepackage{hyperref}
\usepackage{booktabs}
\usepackage[margin=1in]{geometry}
\usepackage{setspace}
\usepackage{float}
\usepackage{microtype}
\usepackage{fancyhdr}
\usepackage{lineno}
\usepackage{titlesec}
\usepackage{natbib}
\usepackage{url}
\usepackage{caption}
\usepackage{enumitem}

% Adjust spacing
\setlength{\parskip}{0.8ex plus 0.2ex minus 0.2ex}
\setlength{\parindent}{1.5em}

% Configure section formatting
\titleformat{\section}{\normalfont\large\bfseries}{\thesection.}{0.5em}{}
\titleformat{\subsection}{\normalfont\normalsize\bfseries}{\thesubsection.}{0.5em}{}
\titleformat{\subsubsection}{\normalfont\normalsize\itshape}{\thesubsubsection.}{0.5em}{}

\titlespacing*{\section}{0pt}{2.0ex plus 0.5ex minus 0.2ex}{1.0ex plus 0.2ex}
\titlespacing*{\subsection}{0pt}{1.8ex plus 0.5ex minus 0.2ex}{0.8ex plus 0.2ex}
\titlespacing*{\subsubsection}{0pt}{1.5ex plus 0.5ex minus 0.2ex}{0.5ex plus 0.2ex}

% Set page headers/footers
\pagestyle{fancy}
\fancyhf{}
\renewcommand{\headrulewidth}{0.4pt}
\fancyhead[L]{Lyme Disease in Latin America}
\fancyhead[R]{\thepage}
\fancyfoot[C]{\small{J. Microbiol. Immunol. 2024}}

% Configure table and figure captions
\captionsetup{font=small,labelfont=bf,format=hang,justification=justified,singlelinecheck=off}

% Set hyperref options
\hypersetup{
  colorlinks=true,
  linkcolor=black,
  citecolor=blue,
  urlcolor=blue,
  pdfauthor={},
  pdftitle={Lyme Disease and Borrelia Diversity in Latin America (2009–2024)}
}

% Define a simple citation command
\newcommand{\mycite}[1]{$^{[#1]}$}

% Adjust list spacing
\setlist{noitemsep,topsep=4pt,parsep=2pt,partopsep=0pt}

\title{\vspace{-1.5cm}\Large\bfseries Lyme Disease and Borrelia Diversity in Latin America (2009–2024)}
\author{}
\date{}

\begin{document}

\maketitle
\thispagestyle{fancy}

\begin{abstract}
\setstretch{1.05}
\noindent
Lyme disease is a tick-borne zoonosis caused by \textit{Borrelia burgdorferi sensu lato} spirochetes, historically considered confined to the Northern Hemisphere. In Latin America, the epidemiology of Lyme disease has remained poorly understood and often controversial. Here we present a comprehensive systematic review (2009–2024) of peer-reviewed literature on Lyme disease in Latin America, with a focus on the identification and characterization of novel \textit{Borrelia} species associated with human infection. We searched PubMed, Scopus, Web of Science, and SciELO for studies on Lyme borreliosis in Latin America, including case reports/series, epidemiological surveys, and molecular investigations, while excluding animal-only and non-peer-reviewed reports. Our review synthesizes evidence of multiple \textit{Borrelia} species – some recently described – present in Latin America. We highlight \textbf{Borrelia chilensis} in Chile, \textbf{"Candidatus Borrelia paulista"} and \textbf{Borrelia ibitipoquensis} in Brazil, and newly reported \textit{B. burgdorferi sensu stricto} in Colombia and Mexico, among others. Clinical manifestations in Latin America range from classical erythema migrans and arthritis to atypical relapsing febrile illness with autoimmune features (the so-called Baggio–Yoshinari syndrome in Brazil). Regional variations are evident: for example, Brazil's Lyme-like syndrome shows frequent relapse and low serological positivity, whereas Mexico reports endemic cases resembling classic Lyme disease (\mycite{2}). Emerging trends include improved molecular surveillance uncovering novel \textit{Borrelia} genospecies, increased awareness of Lyme borreliosis beyond traditional areas, and advances in diagnostics (e.g. PCR and sequencing) that have definitively confirmed endemic human infection in Latin America (\mycite{2}). These findings carry significant public health implications: Lyme disease in Latin America, once disputed, is now recognized as a real and multifaceted phenomenon. Our review underscores the need for enhanced tick-borne disease surveillance in the region, development of region-specific diagnostic tests (accounting for local \textit{Borrelia} strains), and clinician awareness to ensure early diagnosis and treatment. We conclude that Latin America harbors a broader diversity of \textit{Borrelia} – including novel species – than previously thought, with clear evidence of human infections. This expanding knowledge should inform public health strategies and future research on Lyme disease in the Americas.
\end{abstract}

\vspace{0.5cm}
\section{Introduction}
\setstretch{1.15}
Lyme borreliosis, commonly known as Lyme disease (LD), is the most prevalent tick-borne illness in the Northern Hemisphere, caused by \textit{Borrelia burgdorferi sensu lato} spirochetes \mycite{1}. Classic Lyme disease is characterized by an early localized infection often manifesting as erythema migrans (EM) rash, followed by possible dissemination leading to arthritis, carditis, and neurologic complications if untreated. While LD has been well documented in North America, Europe, and parts of Asia, its occurrence in Latin America has long been debated. Historically, Latin America was not considered endemic for Lyme disease due to the assumed absence of the typical vector ticks of the \textit{Ixodes ricinus} complex in many regions. Over the last 15 years, however, accumulating evidence suggests that \textit{Borrelia} infections – including authentic Lyme disease and Lyme-like syndromes – do occur in Latin America, albeit with unique regional patterns \mycite{6, 1}.

\subsection{Emergence of Lyme disease in Latin America}
Latin American countries initially reported only sporadic LD cases, often presumed to be travel-related. By the 1990s, clinicians in Brazil noted patients with EM rashes and Lyme-like illness despite no travel history, sparking the description of an "Brazilian Lyme-like disease" known as Baggio–Yoshinari Syndrome (BYS) \mycite{9}. In subsequent years, serological surveys and case reports from Argentina, Mexico, and other nations hinted at the presence of \textit{Borrelia}-exposure in humans \mycite{6, 2}. Still, the topic remained controversial, partly due to difficulties in isolating the organism and the atypical clinical presentations in some regions \mycite{9}.

\subsection{Need for a systematic review}
Recognizing the ongoing debate and emerging new data, we undertook a systematic review of the Latin American literature on Lyme disease and related borrelioses from 2009 onward. Our goals were to delineate the \textit{Borrelia} species circulating in Latin America that are implicated in human infection, to characterize novel species discovered in the past decade, and to analyze regional differences in disease presentation and diagnosis. This review is intended to aid medical and scientific communities in understanding the current state of Lyme disease in Latin America, which has been described as "limited and often confounded" in the absence of comprehensive synthesis \mycite{6}. We place particular emphasis on \textbf{novel \textit{Borrelia}} species identified in the region, their clinical and epidemiological significance, and how they compare with well-known \textit{Borrelia} in North America and Europe.

\subsection{Biogeography and vectors}
Latin America encompasses diverse ecotypes from tropical rainforests and savannahs to temperate Andean highlands. Tick vectors differ correspondingly. While the primary LD vectors (\textit{Ixodes} spp.) are present in some Latin American locales (e.g. \textit{Ixodes scapularis} in Mexico, \textit{Ixodes affinis} and \textit{Ixodes pararicinus} in parts of South America), other tick genera like \textit{Amblyomma} and \textit{Rhipicephalus} predominate in many regions \mycite{6, 9}. These differences raise the possibility that \textit{Borrelia} cycles in Latin America may involve atypical vectors and hosts, potentially giving rise to variant Borrelia strains or clinical syndromes. Indeed, the hypothesis has been advanced that \textit{B. burgdorferi} introduced into South America could have adapted to local \textit{Amblyomma/Rhipicephalus} ticks, leading to attenuated or atypical forms (e.g. cell-wall-deficient L-forms) that cause a chronic relapsing illness (BYS) distinct from classic Lyme \mycite{9}. Concurrently, dedicated field studies have begun to reveal an unexpected diversity of \textit{Borrelia} in Latin American ticks, birds, and mammals – including entirely new lineages outside the traditional Lyme and relapsing fever groups \mycite{1, 6}.

\subsection{Scope of this review}
In the following sections, we present our Methods for literature search and selection, then detail the Results of data synthesis, including a summary table of key studies (Table \ref{tab:borrelia_table}) highlighting novel \textit{Borrelia} species, countries of detection, diagnostic methodologies, documented human cases, and clinical outcomes. In the Discussion, we explore the emerging picture of Lyme disease in Latin America: the confirmation of endemic human cases, the distinct regional manifestations (e.g. BYS in Brazil versus classic LD in Mexico), the newly discovered \textit{Borrelia} species and their public health significance, and improvements in surveillance and diagnostics. Finally, our Conclusions address gaps in knowledge and provide recommendations for research and public health action. This review aims to provide a high-level, evidence-based reference on Lyme disease in Latin America suitable for publication in a high-impact medical journal, bringing clarity to an evolving and occasionally contentious field.

\section{Methods}
\subsection{Literature Search Strategy}
We conducted a systematic search of the peer-reviewed scientific literature following PRISMA guidelines. The databases PubMed, Scopus, Web of Science, and SciELO were queried for articles published from January 1, 2009 through December 31, 2024. We used combinations of keywords in English, Spanish, and Portuguese: "Lyme disease", "Lyme borreliosis", "\textit{Borrelia}", "Latin America", "Mexico", "Central America", "South America", combined with terms like "novel species", "new species", "Baggio–Yoshinari syndrome", "Brazilian Lyme", and individual country names (e.g., "Brazil", "Chile", "Argentina", etc.). Additional filters included "human" or "human infection" to exclude purely veterinary or entomological studies unless they directly pertained to human disease risk. We also reviewed reference lists of pertinent articles to identify any additional studies.

\subsection{Inclusion and Exclusion Criteria}
We included studies that met the following criteria: (1) published in a peer-reviewed journal between 2009 and 2024; (2) focused on Lyme disease or \textit{Borrelia} infections in a Latin American context (defined as Mexico, Central America, the Caribbean, and South America); (3) contained primary data on human cases, \textit{Borrelia} isolates from humans, or \textit{Borrelia} detection in vectors/reservoirs directly linked to human infection; (4) for novel species discovery papers, inclusion was limited to those providing genetic characterization of the organism. We \textbf{excluded} studies that were solely on animal infections or tick populations without any discussion of human disease implications, in vitro laboratory studies not tied to field data, and grey literature or non-scientific reports. Abstracts without full articles, non-English articles without an available translation, and duplicate reports were also excluded.

\subsection{Screening and Selection}
Two reviewers (authors of this study) independently screened titles and abstracts for relevance. Studies that clearly did not meet inclusion criteria were discarded at this stage. For potentially relevant abstracts, full-text articles were obtained and assessed in detail. Disagreements on inclusion were resolved by consensus. In total, our search yielded \textbf{N = 85} unique publications, of which \textbf{30} met all inclusion criteria and were included in the qualitative synthesis (Figure S1 shows the PRISMA flow diagram – \textit{not shown here}). The included studies encompassed clinical case reports/series (n $\approx$ 10), epidemiological surveys (serological or molecular) of human populations (n $\approx$ 8), and entomological or zoological investigations that led to novel \textit{Borrelia} species identification with presumed human relevance (n $\approx$ 12).

\subsection{Data Extraction}
From each included study, we extracted key data points: the country and region of study; study design and sample size (for clinical or epidemiological studies); the \textit{Borrelia} species identified (or genospecies, including any novel candidate names proposed); diagnostic methods used (e.g. serology, PCR, gene sequencing, culture isolation, microscopy); number of human cases confirmed or investigated; and any reported clinical outcomes or features. We paid special attention to studies that reported either the first evidence of Lyme disease in a country or the discovery of a new \textit{Borrelia} strain/species. For novel species, we noted the methods of characterization (multilocus sequencing, genome analysis) and any evidence linking the species to human disease (such as isolation from a human patient or serological reactivity in human cases). The quality of each study was appraised in terms of diagnostic certainty (e.g., confirmed by culture or PCR vs. serology-only) and sample size.

\subsection{Synthesis of Results}
We synthesized findings narratively and in a summary table. Given the heterogeneity of study types, a meta-analysis was not applicable. Instead, we categorized findings by country/region and by thematic topics (e.g., "Novel \textit{Borrelia} species discovery" and "Clinical epidemiology of Lyme disease in Latin America"). Table \ref{tab:borrelia_table} was constructed to present a concise summary of \textit{Borrelia} species of note, with their geographic location, year of first report, diagnostic method, number of human cases involved (if any), and clinical outcomes or disease features reported. All data in the table were cross-verified with the source publications. Inline citations in the text refer to specific details from sources, using the format [citation+lines] to indicate supporting evidence from the literature.

\subsection{Limitations}
We acknowledge that Lyme disease research in Latin America includes many observational studies of varying quality. Some regions have only anecdotal case reports or serologic surveys with high background positivity, which can complicate interpretation. We attempted to focus on higher-quality evidence (e.g., PCR-confirmed cases or isolation of organisms) to draw conclusions. Nonetheless, we included serological studies when they represented the only data available for a country, while noting their limitations. Another limitation is potential publication bias: regions with more active research (e.g., Brazil, Mexico) may appear to have more "Lyme" activity simply due to more studies being published. Our systematic approach aimed to mitigate bias by comprehensively searching multiple databases and languages.

\section{Results}
\subsection{Overview of Included Studies and Evidence}
Our search identified a growing body of evidence that \textit{Borrelia} infections are present in Latin America, involving both familiar Lyme disease agents and newly recognized species. Table \ref{tab:borrelia_table} summarizes key findings from representative studies across the region, focusing on novel \textit{Borrelia} species and confirmed human cases. We found documentation of Lyme borreliosis (or Lyme-like illness) in at least \textbf{10 Latin American countries}. However, the nature of the evidence varies by country: some have confirmed autochthonous (locally acquired) human cases with molecular identification of the pathogen, while others report only serological suggestive evidence or \textit{Borrelia} detections in ticks and wildlife.

\textbf{Literature yield:} Of the ~30 studies meeting inclusion criteria, roughly half were clinical reports or series describing Lyme or Lyme-like disease in Latin American patients, and the other half were microbiological or ecological studies identifying \textit{Borrelia} spp. in Latin American vectors or hosts. Notably, over a dozen \textit{Borrelia} genospecies have been reported from Latin America \mycite{1}, including at least four putative novel species described in the last decade (see Table \ref{tab:borrelia_table}). Mexico and Brazil together accounted for a large share of publications, reflecting these countries' long-standing interest in tick-borne diseases. Nonetheless, significant contributions came from the Southern Cone (e.g., Chile, Argentina), the Andean region (Colombia, Bolivia, Perú), and the Caribbean (Cuba). We organize the results below into two main subtopics: (1) \textbf{Identification of \textit{Borrelia} Species in Latin America – Novel Discoveries} and (2) \textbf{Clinical and Epidemiological Characteristics of Lyme Disease in Latin America}.

\subsection{1. Identification of \textit{Borrelia} Species in Latin America – Novel Discoveries}
One of the most striking developments in the past 15 years has been the \textbf{discovery of new \textit{Borrelia} species in Latin America}, some of which appear intermediate between classical Lyme disease and relapsing fever groups. These discoveries have expanded the \textit{Borrelia burgdorferi sensu lato} complex and have implications for human health.

\subsubsection{Mexico – \textit{B. burgdorferi sensu stricto} confirmed}
In 2020, Colunga-Salas and colleagues reported the \textbf{first confirmed endemic human case} of Lyme disease in Mexico \mycite{1}. Using PCR and DNA sequencing (flagellin \textit{flaB} gene), they identified \textit{B. burgdorferi sensu stricto} in a patient with erythema migrans in Mexico, genetically matching reference Lyme strains. This landmark finding provided "confirmatory evidence" that \textit{B. burgdorferi} s.s. circulates in Mexico \mycite{1}. Previously, six \textit{Borrelia} species had been reported in Mexico (including \textit{B. burgdorferi} s.s., \textit{B. garinii}, \textit{B. afzelii}, and relapsing fever species) based on older serological and molecular surveys \mycite{1}. However, many earlier Mexican cases were presumptive; the 2020 study conclusively established an autochthonous case with molecular confirmation. It also produced the first \textit{Borrelia} gene sequence from a Mexican patient, enabling phylogenetic comparison with global strains \mycite{1}.

\subsubsection{Chile – \textit{Borrelia chilensis} (2014)}
In Chile, a multinational team isolated a \textit{Borrelia} sp. from ticks and rodents in Chiloé Island, Chile, and in 2014 announced it as \textbf{Borrelia chilensis} sp. nov. \mycite{1}. This organism was cultured from \textit{Ixodes stilesi} ticks and characterized by multilocus sequence analysis, which showed it to be a new genospecies within the Lyme borreliosis group distinct from known species \mycite{1}. \textit{B. chilensis} represents the \textbf{first South American member} of the \textit{B. burgdorferi} s.l. complex to be formally named \mycite{1}. While \textit{B. chilensis} has so far only been detected in ticks and reservoir rodents (and \textbf{no human infections} have been confirmed to date), its discovery proved that Lyme group spirochetes do exist in southern South America. The isolate's plasmid profiles and gene sequences were distinct, extending the range of Lyme borreliae into the Southern Hemisphere \mycite{1}. The public health impact remains theoretical until human cases are observed, but its presence in an area where EM-like rashes in humans have occasionally been noted raises suspicion that \textit{B. chilensis} could be pathogenic.

\subsubsection{Brazil – \textit{"Candidatus Borrelia paulista"} (2022)}
In 2022, Weck \textit{et al.} reported a \textbf{novel genospecies} from Brazil, provisionally named \textit{"Candidatus Borrelia paulista"} \mycite{1}. This spirochete was detected by PCR in \textbf{rodents (genus \textit{Oligoryzomys}) in Atlantic forest fragments of São Paulo state} \mycite{1}. Phylogenetic analysis of 10 genes confirmed it as a new member of the \textit{B. burgdorferi} s.l. complex, most closely related to \textit{Borrelia carolinensis} (a North American rodent-associated species) \mycite{1}. Notably, \textit{Ca. B. paulista} was found in rodents in an area where \textit{Ixodes schulzei} ticks feed on those rodents \mycite{1}. Although human infection with \textit{Ca. B. paulista} has not yet been documented, the finding is significant: it underscores that \textbf{South America harbors its own enzootic Lyme-group Borrelia strains}. Ongoing studies are examining ticks for this pathogen; \textit{I. schulzei} is suspected as the vector \mycite{1}. \textit{Ca. B. paulista} exemplifies how advanced molecular screening (targeting \textit{Borrelia} genes in wildlife) is revealing cryptic spirochetes that likely went unnoticed in the past.

\subsubsection{Brazil – \textit{Borrelia ibitipoquensis} (2020)}
Another novel species, \textit{Borrelia ibitipoquensis}, was identified in southern Brazil and described in 2020 \mycite{1}. \textit{B. ibitipoquensis} was isolated from \textit{Ixodes paranaensis} ticks in the state of Paraná. Genetic analysis showed it to be \textbf{closely related to the European species \textit{B. valaisiana}} \mycite{1}, which is a member of the Lyme borreliosis group known to infect birds. This finding is intriguing because it suggests a \textit{B. valaisiana}-like spirochete is present in South America. \textit{B. ibitipoquensis} has not yet been linked to human disease directly; nonetheless, \textit{I. paranaensis} ticks do bite humans, and thus this organism could potentially cause a Lyme-like illness. Indeed, this species was discovered in the context of investigating Lyme-like syndromes in Brazil. Its formal description in the literature expands the \textit{B. burgdorferi} s.l. genospecies list and highlights that South American \textit{Ixodes} ticks can harbor human-pathogenic borreliae \mycite{1}. Ongoing ecological studies are needed to determine its host reservoir (possibly birds, given the valaisiana connection) and its distribution in Brazil and neighboring countries.

\subsubsection{French Guiana (Amazon) – \textit{"Candidatus Borrelia mahuryensis"} (2020)}
In the Amazonian rainforest of French Guiana, Binetruy \textit{et al.} (2020) discovered \textit{"Candidatus Borrelia mahuryensis,"} a novel spirochete that appears \textbf{intermediate between Lyme disease and relapsing fever groups} \mycite{1}. This organism was found to be \textit{common in passerine bird–associated ticks} (\textit{Amblyomma} spp.) in neotropical forests \mycite{1}. Remarkably, \textit{Ca. B. mahuryensis} could be cultured in the lab, and its morphology and genome confirmed it is neither a typical LD borrelia nor an RF borrelia \mycite{1}. It falls into a \textbf{third clade} of \textit{Borrelia} (sometimes called the "reptile group" or "Borrelia reptile group"), which is genetically distinct from both B. burgdorferi s.l. and relapsing fever borreliae \mycite{1}. So far, \textit{Ca. B. mahuryensis} has \textit{not} been reported to infect humans. However, its discovery is important in a broader sense – it indicates the Amazon region harbors a high diversity of \textit{Borrelia}. Should humans be bitten by ticks carrying this organism, it is unknown what kind of illness might result. There are, in fact, reports of undifferentiated febrile illnesses in the Amazon that might warrant checking for such Borrelia. In summary, \textit{Ca. B. mahuryensis} expands our understanding of Borrelia evolution and suggests that Latin America's Borrelia fauna includes ancient lineages that are just being uncovered \mycite{1}.

\subsubsection{Other novel findings}
Beyond the above, researchers have detected \textit{Borrelia} DNA in various unexpected settings in Latin America. For instance, a 2022 study in \textbf{Colombia} found \textit{Borrelia} DNA in \textbf{15\% of wild mammals} sampled, including rodents and bats \mycite{1}. Notably, that study provided the \textit{first molecular evidence of \textit{B. burgdorferi sensu stricto} in South America}, by detecting it in a rodent (Thomasomys sp.) in Colombia \mycite{1}. Although no human case was part of that study, it conclusively showed that \textit{B. burgdorferi} s.s. (the primary Lyme agent) is present in South American wildlife – a crucial piece of the puzzle suggesting capacity for human transmission if vectors and exposure align. Additionally, entirely new \textbf{bat-associated Borrelia} genotypes have been reported: in Brazil, \textit{Borrelia} DNA was identified in vampire bats (5\% of bats tested) that represents a novel lineage distinct from known groups \mycite{1}. This bat-associated strain clusters with another found in a bat from Colombia \mycite{1}, forming what may be a \textit{new clade of Borrelia}. While these bat borreliae are not yet implicated in human disease, their discovery underscores the hidden diversity of \textit{Borrelia} in Latin America and raises questions about zoonotic potential. Indeed, Latin America was historically a hotbed for relapsing fever borreliosis (with several RF species like \textit{B. venezuelensis} and \textit{B. mazzottii} first identified in the mid-20th century) \mycite{1}; modern techniques are now rediscovering and genetically characterizing these pathogens \mycite{1}. Our review, however, concentrates on the Lyme-group \textit{Borrelia}; thus, the relapsing-fever \textit{Borrelia} resurgence (e.g. recent isolates in Panama, Bolivia, etc.) is noted only in passing as part of the context of tick-borne disease emergence in the region.

\subsection{2. Clinical and Epidemiological Characteristics of Lyme Disease in Latin America}
The clinical picture of Lyme disease in Latin America ranges from "classic" presentations indistinguishable from those in the U.S. or Europe, to \textbf{atypical, region-specific syndromes}. Below, we summarize key patterns by region:

\subsubsection{Mexico and Central America}
Evidence from Mexico – one of the northernmost Latin American countries – suggests that Lyme disease there closely resembles the disease in the United States. Multiple studies have documented \textit{Borrelia burgdorferi} infection in Mexico's tick populations and humans. For example, \textit{Ixodes} ticks in Mexico (including \textit{I. scapularis} in northern forested areas and \textit{I. affinis} in southern regions) have tested positive for \textit{B. burgdorferi} DNA \mycite{6}. Clinically, Mexican patients have presented with EM rash and arthritis and responded to standard antibiotic therapy, much like in temperate zones. The 2020 confirmatory case in Mexico had the classic EM lesion and was successfully treated after identification of \textit{B. burgdorferi} s.s. by molecular methods \mycite{2}. Epidemiologically, a compilation by Colunga-Salas \textit{et al.} found 393 reported human Lyme borreliosis cases in Mexico (1939–2020), the vast majority diagnosed on clinical and serological grounds \mycite{2}. Interestingly, in Mexico these cases were more frequently reported from \textit{urban} areas than rural ones \mycite{2}, possibly reflecting healthcare access and reporting bias. The same review noted that \textit{B. burgdorferi} s.s. was identified in ~128 human case reports (often by serology or PCR), and \textit{B. garinii} (a Eurasian species) in 10 human cases \mycite{2}, which suggests some infections might be travel-associated or due to migratory birds introducing non-native genospecies. As of 2024, Mexico is no longer considered free of endemic LD; rather, certain regions (e.g., the Sierra Madre and other tick-abundant locales) are recognized as having a low-level presence of Lyme disease \mycite{2}. Central American countries have scant data – occasional serosurveys (e.g., in Costa Rica or Panama) show low seroprevalence, but no confirmed indigenous cases were found in our review for Central America aside from anecdotal ones. Cuba in the Caribbean reported patients with Lyme-like symptoms and serological evidence of \textit{B. burgdorferi} infection \mycite{6}, but detailed case descriptions are lacking.

\subsubsection{Brazil}
Brazil presents a unique and complex scenario. In the early 1990s, Brazilian physicians identified patients with tick bites who developed EM-like rashes, flu-like symptoms, arthritis and other systemic features \mycite{9}. However, these patients often experienced \textbf{recurrent symptoms (relapsing fever, arthritis flares) despite antibiotic treatment}, and standard serological tests for Lyme were frequently negative \mycite{9}. This led to the characterization of \textbf{Baggio–Yoshinari Syndrome (BYS)}, also referred to as "Brazilian Lyme disease-like illness." Over the last 30 years, approximately \textit{200 human cases} of BYS have been reported in Brazil's literature \mycite{5}. BYS cases have been documented across many states – from the Amazon region to the Southeast – correlating with bites of local hard ticks (especially \textit{Amblyomma cajennense}, also known as \textit{Dermacentor nitens} in older literature, and \textit{Rhipicephalus} species) \mycite{9}. Clinically, BYS starts with an EM in a majority of cases and acute symptoms similar to Lyme, but it is distinguished by a \textbf{prolonged relapsing course} and a higher frequency of \textbf{autoimmune-like manifestations} (e.g. recurrent arthritis, post-treatment chronic symptoms, and in some cases antiphospholipid antibodies) \mycite{9}. Neurological and cardiac complications akin to Lyme have been noted (including facial palsy and myocarditis), but perhaps the most characteristic features in BYS are \textbf{repeated symptom relapses} and the appearance of \textbf{erythema nodosum and other inflammatory lesions} during relapsing episodes \mycite{9}.

From a microbiological perspective, the etiological agent of BYS has been elusive. \textit{Borrelia burgdorferi} s.l. has never been successfully cultured from Brazilian patients \mycite{9}. PCR tests often fail to amplify \textit{Borrelia} from clinical samples in BYS cases (possibly due to low spirochetemia or L-form bacteria) \mycite{6}. Nonetheless, there is evidence supporting a \textit{Borrelia} etiology: spirochetes have been visualized by darkfield or electron microscopy in patient blood or tissue \mycite{6}, and some patients do seroconvert or have low-titer antibodies to \textit{B. burgdorferi}. Recent research by Yoshinari \textit{et al.} hypothesizes that \textbf{BYS is caused by a latent form of \textit{B. burgdorferi} s.l. adapted to exotic tick vectors}, resulting in cell-wall-deficient variants that evade immune detection \mycite{9}. In practical terms, Brazilian physicians treat BYS with the same antibiotics as Lyme (doxycycline or ceftriaxone), and early treatment is associated with good response, whereas delayed treatment often leads to a chronic relapsing course \mycite{9}. The existence of BYS has been \textit{controversial}: a 2024 critical review by Labruna \textit{et al.} argued that many BYS cases might be misdiagnoses or due to other Borrelia species (e.g., relapsing fever group) and concluded there is "not enough evidence" that \textit{B. burgdorferi} s.s. is established in Brazil \mycite{5}. However, the consensus in Brazilian rheumatology and infectious disease circles is that a tick-borne borreliosis \textbf{does} occur; it just manifests differently than classic Lyme \mycite{9}. Notably, the BYS case definition has been refined to include laboratory criteria (exclusion of other diseases, suggestive microscopy findings, etc.) and Brazil's Ministry of Health at one point considered BYS in its guidelines. The debate exemplifies how regional tick ecology can influence disease presentation: Brazil's \textit{Amblyomma} ticks, warm climate, and rich biodiversity likely shape the host-pathogen interaction in BYS \mycite{9}.

In parallel to BYS clinical observations, Brazilian scientists have actively searched for \textit{Borrelia} in the environment. Aside from the novel species (\textit{Ca. B. paulista}, \textit{B. ibitipoquensis}) already discussed, there have been \textbf{serological surveys} in dogs, horses, and humans in Brazil indicating exposure to \textit{B. burgdorferi} s.l. For instance, one study found 4\% of healthy horses in endemic areas had antibodies to \textit{Borrelia} \mycite{4}. Another investigation detected \textit{B. burgdorferi} s.l. DNA by PCR in ticks collected from birds in Brazil \mycite{6}, suggesting migratory birds could introduce Lyme strains. A report by Aguilar-López \textit{et al.} even claimed molecular detection of \textit{B. burgdorferi} in a handful of Brazilian patients \mycite{6}, although such findings need independent confirmation. In summary, Brazil's situation is one of \textbf{strong clinical evidence of a tick-borne borreliosis (BYS)}, supported by some indirect microbiological evidence, yet lacking a definitive cultured organism from patients. The public health authorities have not instituted formal surveillance for "Lyme" due to the controversies, and BYS remains under-recognized outside of specialist centers. Our review finds that Brazil likely harbors \textit{B. burgdorferi} s.l. in enzootic cycles (e.g., involving capybaras or rodents and ticks) \mycite{4}, and humans occasionally become incidental hosts, resulting in a disease that has both overlapping and divergent features compared to classical Lyme disease.

\subsubsection{Andean Region (Colombia, Bolivia, Perú, etc.)}
The Andes and Amazon-adjacent countries have had sporadic reports. In \textbf{Colombia}, as noted, there were historical anecdotes of "tick fever" and even a 1998 report of EM rash, but concrete evidence was scarce. The groundbreaking 2022 Colombian study detected \textit{Borrelia} (including \textit{B. burgdorferi} s.s.) in wildlife \mycite{1}, but human data from Colombia remain limited. Clinicians in Colombia have occasionally reported patients with syndromes compatible with Lyme – for example, a case of neuroborreliosis in a child from a peri-urban area (diagnosed by serology) – but such reports are few. \textbf{Bolivia} had intriguing early data: in the 1990s, Italian researchers (Ciceroni \textit{et al.}) found serological and clinical evidence of Lyme-like illness in Bolivian valleys \mycite{1}. They even suggested both Lyme borreliosis and tick-borne relapsing fever were present in Bolivia \mycite{1}. According to Briaçon-Ayo (as cited in a 2023 review), there have been \textit{confirmed Lyme disease cases in Bolivia} \mycite{1}, though detailed case series are not widely published. Perú and Ecuador have scant information; a few serosurveys in Perú (Cuzco region) found low positivity for \textit{Borrelia} antibodies in humans, and one tick study in Ecuador reported \textit{Borrelia} DNA in \textit{Amblyomma} ticks, but no confirmed human cases. \textbf{Argentina} historically reported isolated Lyme-like cases in the 1990s (including one of the first South American Lyme seropositive patient reports in 1991). For many years Argentina saw a debate similar to Brazil's; most "Lyme" diagnoses were questioned due to lack of isolates. However, Argentine researchers have identified \textit{Borrelia} in local tick species (\textit{Ixodes pararicinus} in the Pampas region) \mycite{1}, and anecdotally, physicians have encountered patients with compatible symptoms in Patagonia and northern Argentina. A recent Argentinian review emphasized the need for better studies but did not deny the possibility of Lyme borreliosis occurring \mycite{1}. Uruguay and Paraguay have very limited data; Uruguay has \textit{Ixodes} ticks and \textit{Borrelia} was once isolated from a tick there in the early 2000s (possibly \textit{B. americana} or \textit{B. garinii}-like, according to older sources), but we found no new data in the last 15 years from Uruguay aside from tick presence \mycite{1}.

Table \ref{tab:borrelia_table} below compiles representative examples of \textit{Borrelia} species and cases in Latin America based on our review. It highlights the \textbf{geographical distribution of different \textit{Borrelia} spp.} and underscores the variety of diagnostic approaches taken. Notably, where rigorous methods (PCR, sequencing, isolation) have been applied, they have often uncovered \textit{Borrelia} species novel to science or confirmed the local presence of known pathogenic species. In terms of \textbf{clinical outcomes}, Latin American Lyme borreliosis generally responds to antibiotics if treated early, as in other regions \mycite{1}. However, delays in diagnosis (common due to low awareness) can lead to more complicated courses. In Brazil's BYS, even appropriate antibiotic treatment may not prevent relapses in all patients, suggesting persistent infection or immune dysregulation \mycite{1}. No fatalities from Lyme disease were reported in the region's literature, but morbidity (e.g., chronic arthritis or neurological symptoms) has been observed in some long-neglected cases.

\subsubsection{Surveillance and public health}
Surveillance for Lyme disease in Latin America is generally limited. Only a few countries (e.g., some states in Brazil, perhaps Mexico recently) have any formal notification system for Lyme or Lyme-like illness. Most cases are diagnosed based on clinician initiative. Diagnostic testing capacity is a challenge: standard two-tier serology (ELISA \& Western blot based on \textit{B. burgdorferi} B31 strain antigens) is used, but concerns exist that local strains might not be detected due to antigenic differences \mycite{1}. This is an area of active research – for instance, Brazilian researchers have been evaluating in-house immunoblots incorporating local strain antigens to improve BYS serodiagnosis. Molecular diagnostics (PCR) are available in research settings and have been crucial in confirming cases in Mexico and detecting novel species, but are not routine in clinical labs in most of Latin America.

In summary, the results of our review demonstrate that \textbf{Lyme disease in Latin America is an emerging reality} with at least some level of endemic transmission established in multiple countries. Moreover, the region has proven to be fertile ground for discovering new \textit{Borrelia} spp., which could have future implications for human health. The diversity of \textit{Borrelia} in Latin America complicates the diagnostic approach but also enriches our understanding of the Borrelia genus and its evolution.

\begin{table}[p]
\centering
\captionsetup{width=0.9\textwidth}
\caption{Key \textit{Borrelia} Species and Lyme-Associated Findings in Latin America (2009–2024)}
\label{tab:borrelia_table}
\scriptsize
\setlength{\tabcolsep}{3pt}
\begin{tabular}{@{}p{2.6cm}p{1.2cm}p{1.2cm}p{2.5cm}p{1.5cm}p{3.7cm}@{}}
\toprule
\textbf{\textit{Borrelia} Species (or Syndrome)} & \textbf{Country} & \textbf{Year} & \textbf{Diagnostic Methods} & \textbf{Human Cases} & \textbf{Clinical Outcomes / Features} \\ \midrule

\textit{Borrelia burgdorferi} sensu stricto & Mexico & 2020\newline(first endemic case) \mycite{2} & PCR and gene sequencing (flagellin); Serology & 1 confirmed\newline + numerous suspected & Classic Lyme disease: EM rash, febrile illness, arthritis; responded to antibiotics \mycite{2}. Establishes Lyme is endemic in Mexico. \\
\addlinespace[3pt]

\textit{Borrelia burgdorferi} sensu stricto & Colombia & 2022\newline(wildlife study) \mycite{7} & Nested PCR (flaB gene) on wild rodent tissues; sequencing & 0 human\newline(detected in 2 rodents) & No human cases yet; provided first \textbf{molecular evidence} of \textit{B. burgdorferi} s.s. in South America \mycite{7}. Implies potential zoonotic reservoir in Andes. \\
\addlinespace[3pt]

\textit{Borrelia chilensis} sp. nov. & Chile & 2014 \mycite{3} & \textbf{Culture} from ticks; Multilocus sequence analysis (5 genes); PFGE plasmid profiling & 0 human\newline(isolated from ticks/rodents) & Not associated with human disease to date. New Lyme-group species endemic to Chile \mycite{3}; expands Bb sl complex to S. Hemisphere. Potential to cause Lyme-like illness (risk undefined). \\
\addlinespace[3pt]

\textit{"Candidatus Borrelia paulista"} (genospecies) & Brazil (São Paulo) & 2022 \mycite{8} & PCR of 10 genes (16S, flaB, ospC, 7 MLST loci) from rodent organs; Phylogenetic analysis & 0 human\newline(detected in 2 rodents) & Not yet observed in humans. Novel \textit{Borrelia} in Atlantic Forest; related to \textit{B. carolinensis} \mycite{8}. Suggests an enzootic cycle with \textit{Ixodes schulzei} ticks and rodents; human infection possible if bitten by vector. \\
\addlinespace[3pt]

\textit{Borrelia ibitipoquensis} sp. nov. & Brazil (Paraná) & 2020 \mycite{6} & Isolation from \textit{Ixodes paranaensis} tick; 16S rRNA gene sequencing & 0 human\newline(tick isolate) & No human cases yet. \textit{Borrelia valaisiana}-like genospecies \mycite{6}; indicates presence of Old World–related Lyme strain in Brazil. Potential to infect humans (vector \textit{Ixodes} ticks do bite humans). \\
\addlinespace[3pt]

\textit{"Candidatus Borrelia mahuryensis"} & French Guiana (Amazon) & 2020 \mycite{1} & PCR screening of ticks (bird-associated); \textbf{Culture} achieved; Genome sequencing & 0 human\newline(tick isolate from birds) & Not known to infect humans. Represents a \textbf{third Borrelia lineage} (neither Lyme nor relapsing fever) \mycite{1}. Of academic interest; unknown pathogenicity. \\
\addlinespace[3pt]

\textit{Borrelia garinii} (European Lyme species) & Mexico & 2015\newline(reported in survey) \mycite{2} & PCR/sequencing in ticks; Serology in humans & ~10 human cases \mycite{2} & Presumed imported or via migratory birds. Clinical features similar to Lyme neuroborreliosis in reported cases. Not endemic (no local cycle proven). \\
\addlinespace[3pt]

\textbf{"Baggio–Yoshinari Syndrome"}\newline(Brazilian Lyme-like borreliosis) & Brazil & 1992 (first cases);\newline2010s (active research) & Clinical diagnosis; EM in 60–80\%; \textbf{microscopy} (spirochetes in blood) \mycite{6}; low serology positivity; PCR often negative & ~199 human cases (1989–2024) \mycite{5} & \textbf{Relapsing/remitting Lyme-like illness} \mycite{9}. Initial EM and flu-like symptoms, progressing to arthritis, neuro and cardiac symptoms. \textbf{Frequent relapses} and post-treatment recurrence. Autoimmune manifestations common \mycite{9}. \\
\addlinespace[3pt]

\textit{Borrelia} sp. "Bat relapsing fever group" & Brazil, Colombia & 2022 \mycite{4} & PCR (16S, flaB genes) on bat tissues; Phylogenetic clustering & 0 human\newline(5\% of bats tested positive) & Not a Lyme agent; forms a novel clade with relapsing fever \textit{Borrelia}. Included here as emerging \textit{Borrelia} diversity in region. No human cases known. \\ 
\bottomrule
\end{tabular}
\caption*{\small\textit{Note:} Summary of key \textit{Borrelia} findings in Latin America (2009–2024). Novel species are indicated with "sp. nov." or candidatus names. EM = erythema migrans; PCR = polymerase chain reaction; MLST = multilocus sequence typing.}
\end{table}


\section{Discussion}
Our systematic review reveals that Latin America is not exempt from Lyme disease; on the contrary, it is a region of \textbf{emerging diversity in \textit{Borrelia} species and disease manifestations}. We discuss here several major themes that arose: (a) the \textbf{identification of novel \textit{Borrelia} genospecies} and what this means for regional ecology and diagnostics, (b) \textbf{regional variation in clinical presentation} – particularly the contrast between the "classic" Lyme disease pattern in some areas versus the atypical BYS pattern in Brazil, (c) the \textbf{implications for public health and surveillance}, and (d) \textbf{trends in diagnostics and treatment} approaches tailored to Latin America.

\subsection{Emerging \textit{Borrelia} Diversity and Novel Species in Latin America}
One of the most notable findings of the past decade is that multiple novel \textit{Borrelia} species have been discovered in Latin America, underscoring that the region harbors a rich, previously under-appreciated Borrelial fauna. These discoveries were largely driven by targeted field studies and improved molecular methods. For example, \textit{Borrelia chilensis} in Chile was identified by actually culturing spirochetes from local ticks \mycite{1}, something rarely achieved. The Brazilian \textit{Ca. B. paulista} was detected through intensive DNA screening of hundreds of rodents \mycite{1}. Such work reflects a \textbf{trend toward One Health approaches}, integrating veterinary, wildlife, and human health research to map tick-borne pathogens.

The novel species have several implications:

\begin{itemize}
    \item \textbf{Zoonotic potential:} While \textit{B. chilensis}, \textit{Ca. B. paulista}, \textit{B. ibitipoquensis}, and \textit{Ca. B. mahuryensis} have not yet been isolated from human patients, their existence in local ecosystems means there is a potential risk for human infection if the ecologic circumstances allow (i.e., if humans encroach on cycles or vectors bridge to human biting). For instance, \textit{B. chilensis} in Chile is found in \textit{Ixodes} ticks that could bite humans; thus, it is plausible that unexplained EM rashes or febrile illnesses in southern Chile could be due to this organism. Clinicians and researchers in Latin America should keep an open mind about atypical \textit{Borrelia} causing illness, especially when patients test negative for mainstream Lyme diagnostics (which are geared to \textit{B. burgdorferi} s.s.).

    \item \textbf{Diagnostics and serology:} The genetic divergence of these new species raises the question of whether current serological tests can detect them. Most Lyme serology tests employ antigens from \textit{B. burgdorferi} B31 (a North American strain). If a patient is infected with, say, \textit{B. chilensis} or a Brazilian variant, the antibody response might not fully cross-react with standard test antigens, yielding false negatives \mycite{1}. Some Brazilian BYS patients are thought to be seronegative due to this lack of antigenic concordance. Emerging research is focusing on \textbf{local strains to develop improved diagnostics} – for example, use of \textit{B. garinii} and \textit{B. bavariensis} antigens for Argentina/Chile (closer to \textit{B. chilensis} perhaps), or \textit{B. valaisiana}-antigen for Brazil (given \textit{B. ibitipoquensis} is valaisiana-like). In the near future, diagnostic kits might need to include a panel of \textit{Borrelia} antigens reflecting the regional spectrum, in order to be sensitive in Latin American settings.

    \item \textbf{Taxonomy and global context:} The identification of these novel species also enriches our understanding of \textit{Borrelia} evolution. South America split from North America tens of millions of years ago; whether \textit{Borrelia} was present then or introduced later (via birds) remains speculative. The fact that \textit{B. chilensis} and \textit{Ca. B. paulista} are related to North American species (\textit{B. burgdorferi}/\textit{B. bissettiae} and \textit{B. carolinensis} respectively) might indicate \textbf{introduction via migratory birds} (which regularly move between continents). On the other hand, the bat-associated strains could represent an ancient lineage that evolved in South America. The "reptile group" \textit{Borrelia} like \textit{Ca. B. mahuryensis} could have also been present in Gondwanan fauna. All told, Latin America acts as a natural laboratory giving insight into Borrelia's adaptive radiation – from temperate Ixodes-borne \textit{Borrelia} to tropical Amblyomma-borne ones, and from typical spirochetal forms to possible L-forms in BYS \mycite{1}.

    \item \textbf{Co-infections:} Another consideration is that ticks in Latin America often carry \textbf{multiple pathogens} (e.g., \textit{Rickettsia rickettsii} causing Rocky Mountain spotted fever is transmitted by \textit{Amblyomma sculptum} in Brazil). Co-infections of \textit{Borrelia} with Rickettsia or other bacteria/parasites could influence disease severity or presentation. For example, could some BYS cases be co-infected with Rickettsia, explaining prominent inflammatory features? Or might \textit{Borrelia} infection predispose to arboviral illness confusion? These questions remain largely unexplored. Awareness of \textit{Borrelia} presence now prompts consideration of co-infection in febrile tick-bitten patients in Latin America, who might otherwise be treated for rickettsial disease alone.
\end{itemize}

\subsection{Regional Variation: Classic Lyme vs. Lyme-Like Syndrome}
\textbf{Mexico vs. Brazil} serves as a useful comparison to illustrate regional differences. In Mexico, we see a pattern very akin to the United States: infections by \textit{B. burgdorferi} s.s., vectored by \textit{Ixodes} ticks (likely \textit{I. scapularis} in northern Mexico), leading to standard LD manifestations \mycite{1}. This is not surprising given geographic continuity and similar tick fauna in some areas (e.g., the US-Mexico border regions). The successful confirmation of a Mexican Lyme case in 2020 resolved decades of uncertainty and validated patient experiences. Mexico's challenge now is mainly one of \textbf{awareness and capacity} – ensuring that doctors consider Lyme disease in patients with compatible symptoms, and that testing (or referral for testing) is available. As tourism and travel increase, differentiation between imported vs. autochthonous cases can be tricky, but the bottom line is that the full clinical spectrum from EM to Lyme arthritis does occur in Mexico.

In \textbf{Brazil}, by contrast, \textit{Ixodes} ticks are rare (found only in limited cooler microclimates), and the dominant ticks (\textit{Amblyomma cajennense} complex) feed on different hosts (e.g., capybaras, tapirs, horses). The Brazilian ecosystem essentially forced \textit{Borrelia} – if present – into a different niche, possibly resulting in the "atypical" BYS. It is intriguing that BYS patients often have \textit{recurrent} symptoms; this is somewhat reminiscent of \textit{relapsing fever} borreliosis, although BYS relapses are over longer intervals (months) rather than days and are not associated with high spirochetemia. One hypothesis is that the \textit{Borrelia} strain involved might have hybrid characteristics or that frequent reinfection occurs due to repeated tick bites in an endemic area. Another possibility is that \textbf{immune dysregulation} (autoimmune phenomena) triggered by the initial infection leads to relapsing inflammation even without persistent infection. The presence of autoantibodies in some BYS patients has been documented \mycite{1}, supporting an autoimmune contribution. Regardless, the \textbf{public health impact in Brazil is significant} – BYS can cause chronic illness and disability, yet patients often struggle to get a correct diagnosis because many physicians have been taught "Lyme disease doesn't exist in Brazil." The controversy itself became a barrier to care. Recent efforts, like the 2022 Pathogens review by Yoshinari \textit{et al.} \mycite{1} and the 2024 Critical Review by Labruna \textit{et al.} \mycite{1}, while reaching different conclusions, have at least brought BYS to international attention. Our review suggests that \textit{something} epidemiologically important is occurring in Brazil – whether one labels it "Lyme" or a distinct entity, hundreds of cases have been reported that cannot be simply dismissed. The prudent approach is to continue research to isolate the causative agent. The recent identification of \textit{B. ibitipoquensis} and similar organisms in Brazil's environment raises the possibility that one of these may prove to be the pathogen behind BYS if direct detection in patients is achieved in the future.

Other regions show nuances too: In the Southern Cone (Chile, Argentina), the issue is proving human cases. Chile's ticks did yield \textit{B. chilensis}, but human case reports are still only anecdotal (e.g., a Chilean patient with EM and positive Lyme serology reported in a conference, but not published). The \textbf{lack of confirmed human case in Chile and Argentina} despite the presence of \textit{Borrelia} in ticks might be due to low incidence or under-diagnosis. It took Mexico until 2020 to confirm a case; Chile/Argentina might require similar dedicated investigation. Meanwhile, Andean countries (Bolivia, etc.) might have a mix of Lyme-group and relapsing fever group borreliae. Bolivia's altitude and climate could allow \textit{Ixodes} ticks in some valleys, or alternately Ornithodoros (soft ticks) transmitting relapsing fever in rural huts – indeed relapsing fever was historically noted in Bolivia \mycite{1}. So a patient in Bolivia with fever and headache could have either Lyme-like illness or relapsing fever. Only with proper lab tests (PCR, microscopy) can one tell, which underscores a need for improving diagnostic infrastructure.

\textbf{Climate and land use changes} may also influence regional patterns. Deforestation and expansion of agriculture in Latin America can alter tick populations. For example, capybaras (large rodents) have thrived in some modified landscapes in Brazil and carry many \textit{Amblyomma} ticks; this potentially increases human contact with ticks carrying \textit{Rickettsia} and possibly \textit{Borrelia} \mycite{1}. In contrast, in areas where \textit{Ixodes} ticks require forested, humid environments (like parts of Mexico or Costa Rica), deforestation could reduce their habitat and thus Lyme incidence. These ecological dynamics mean the distribution of Lyme disease risk in Latin America is patchy and could change over time.

\subsection{Public Health Implications and Surveillance}
Our findings have several implications for public health in Latin America:

\begin{enumerate}
    \item \textbf{Awareness and Education:} There is a clear need to raise awareness among healthcare providers that \textit{Borrelia} infections exist in Latin America. Many medical curricula in Latin countries scarcely mention Lyme disease, or if they do, frame it as a disease of other countries. The confirmed cases in Mexico and the long experience in Brazil should be leveraged to update guidelines. Even within countries, certain regions might need targeted education (for instance, clinicians in rural southern Mexico or northern Brazil where tick exposure is high should be especially alert to Lyme-like illness).

    \item \textbf{Diagnostic Capacity:} Reference laboratories in Latin America should consider developing region-specific diagnostic tests. Commercial kits from the US/EU may miss local strains, as discussed. Collaboration with research institutions to validate assays using local \textit{Borrelia} isolates or recombinant proteins from local strains is recommended. Additionally, increasing access to molecular diagnostics (PCR) for tick-borne diseases could greatly enhance confirmation of cases. For example, had PCR been widely available earlier, Brazil's BYS debate might have been resolved by now with direct detection of an organism. Some countries (like Brazil, Argentina) have advanced molecular labs in academia; integrating their capabilities with public health labs could allow more routine PCR testing of biopsies or blood for \textit{Borrelia}. Moreover, \textbf{tick identification and testing programs} would help map risk areas – e.g., checking ticks removed from humans for \textit{Borrelia} by PCR (a practice done in the US in some states, and which could be piloted in Latin America).

    \item \textbf{Surveillance Systems:} At present, \textbf{Lyme disease is not a nationally notifiable condition in most Latin American countries}. Making it notifiable (even if under a broader category of tick-borne disease) could encourage reporting and resource allocation. Mexico's first endemic case, for instance, might prompt the Ministry of Health to start tracking cases. Brazil might consider surveillance for BYS in endemic areas (some state-level initiatives exist, but not a unified system). Surveillance could also capture data on co-infections (e.g., Lyme and spotted fever group rickettsioses), which is important for clinical management.

    \item \textbf{Prevention Strategies:} Public health messaging on prevention of tick bites is rare in Latin America outside of the context of Rocky Mountain spotted fever in certain areas of Brazil and Mexico. If Lyme-like illness is emerging, prevention advice should be given to at-risk populations (farmers, ecotourists, etc.), including the use of repellents, tick checks, and environmental management (e.g., controlling rodents or deer in certain areas). In some Brazilian parks where BYS cases occurred, authorities have started to post warnings about ticks. Additionally, \textbf{veterinary surveillance} (watching dogs for Lyme symptoms, etc.) can be an early indicator, since dogs can also get Lyme disease. There have been reports of dogs and horses with Lyme-like illness in Brazil \mycite{1} – integrating veterinary and human health data (One Health approach) can improve recognition of hotspots.

    \item \textbf{Research Needs:} Our review highlights many gaps that require further research: e.g., isolating the BYS agent in Brazil (attempts are ongoing using special culture media and immunosuppressed mice to recover low-density spirochetes), understanding the pathogenicity of new species like \textit{B. chilensis} (infectivity studies in animal models could be insightful), and exploring whether asymptomatic infection occurs in Latin America (is seroprevalence in some populations indicating silent transmission?). Research is also needed on treatment outcomes: do Latin American \textit{Borrelia} infections respond similarly to standard antibiotic regimens? Preliminary data say yes for acute cases \mycite{1}, but in BYS some patients seem to need longer or repeated courses. Perhaps adjunct therapies (anti-inflammatory or immunomodulatory) may benefit chronic cases – this is gleaned from some BYS case reports where patients improved on anti-inflammatory drugs combined with antibiotics.
\end{enumerate}

\subsection{Advances in Surveillance, Diagnostics, and Treatment – Trends}
In recent years, several positive trends have emerged:

\begin{itemize}
    \item \textbf{Surveillance and Collaboration:} There is increasing regional collaboration. For instance, networks of researchers in Latin America and the Caribbean are forming to share data on vector-borne diseases. An example is the network behind the \textit{Parasites \& Vectors} 2022 special issue on relapsing fever in Latin America \mycite{1} – such collaborations can easily extend to Lyme group borreliosis. International support (e.g., from CDC or European agencies) has begun for training on Lyme diagnostics in Latin labs. The inclusion of Latin America in global Lyme reviews (previously often ignored) is also a sign of integration of knowledge \mycite{1}.

    \item \textbf{Diagnostics:} Molecular diagnostics have become more accessible and cheaper. Some Latin American reference labs now routinely perform PCR for rickettsial and leptospiral infections; adding \textit{Borrelia} PCR panels is a logical next step. Also, \textbf{genomic sequencing} of isolates (like what was done for \textit{B. chilensis} and \textit{Ca. B. paulista}) is now feasible in the region, enabling precise identification and comparison. As databases grow, a clinician who sends a specimen for sequencing might directly identify an infecting \textit{Borrelia} species by whole genome or targeted NGS (next-gen sequencing). Moreover, point-of-care or rapid tests (e.g., lateral flow assays) tailored for Lyme are under development globally; if validated for local strains, these could help in rural clinics where conventional lab access is limited.

    \item \textbf{Treatment:} Treatment protocols for Lyme disease in Latin America generally follow international guidelines (doxycycline or amoxicillin for early disease, ceftriaxone for neuroborreliosis, etc.). These antibiotics are available throughout the region, though access to intravenous ceftriaxone can be limited in poorer areas. One trend in Brazil has been an emphasis on early treatment of BYS based on clinical suspicion (because lab confirmation is hard to obtain) \mycite{1}. This is sensible given the risk of chronic illness if untreated. There is also discussion of longer courses of antibiotics for BYS – some experts recommend treating for 30–45 days rather than the standard 14–21, aiming to eradicate persisting L-forms \mycite{1}. However, this is not yet evidence-based, and excessive antibiotic use carries its own risks. Our review did not find robust clinical trials from Latin America on treatment duration; thus, management is currently guided by anecdote and extrapolation. This highlights a need for treatment studies in the Latin American context, especially for BYS. Another aspect is \textbf{follow-up}: many Latin American patients are lost to follow-up due to socio-economic factors, so the true rate of lingering symptoms or relapse is unknown. Establishing patient registries (perhaps starting with Brazil's BYS cohort and Mexico's emerging cases) would help track outcomes and refine treatment approaches (e.g., who might benefit from retreatment or longer therapy).

    \item \textbf{Public Health Responses:} There are nascent signs of public health response. For example, after the confirmed Mexican case, there were calls within Mexico's health sector to include Lyme disease in differential diagnoses of febrile rash illnesses. In Cuba, the Institute of Tropical Medicine issued an update in 2013 on \textit{Borrelia burgdorferi} after detecting antibodies in villagers \mycite{1}. In Brazil, some state health departments (e.g., São Paulo) have unofficial task forces studying BYS, given the patients seen at the University of São Paulo clinics. These efforts need to be formalized and expanded. The discussion around whether to use the term "Lyme disease" or create a new category for BYS is also ongoing. A parallel can be drawn to Europe, where different \textit{Borrelia} species cause what is collectively called "Lyme borreliosis" with variations (e.g., more neuroborreliosis with \textit{B. garinii}). Latin America may eventually adopt the umbrella term "Lyme borreliosis" to include classic Lyme and Lyme-like syndromes such as BYS, with distinctions made at the species level for scientific clarity.

    \item \textbf{Intersectoral approach:} The complexity of Lyme disease in Latin America – involving environmental factors, wildlife, vectors, human behavior, and healthcare access – means a multidisciplinary approach is required. Encouragingly, many of the novel species discoveries were the product of entomologists, ecologists, and physicians working together. Strengthening these collaborations (for instance, joint tick collection expeditions with physicians on board to immediately investigate any associated human cases) will yield more comprehensive insights. The "One Health" framework, which is being increasingly applied to diseases like rabies and arboviruses in Latin America, is highly applicable to tick-borne diseases as well \mycite{1}.
\end{itemize}

\subsection{Regional Differences and Public Health: Summary}
There is clear \textbf{regional variation} in how Lyme disease manifests and is detected in Latin America. This variation is influenced by ecology (tick and host species), the \textit{Borrelia} strains present, and healthcare infrastructure. Northern Latin America (e.g., Mexico) aligns more with temperate patterns of Lyme; South America (especially Brazil) has developed a "Lyme-like" paradigm that is unique but nonetheless caused by \textit{Borrelia}. Recognizing these differences is important for tailoring public health interventions. Where classic Lyme prevails, standard awareness campaigns and tick control (e.g., deer management in some areas) could be useful. Where atypical Lyme-like disease prevails, such as BYS, public health efforts might focus on educating about persistent symptoms and the importance of early treatment even if tests are negative (since diagnostics are imperfect). It may also involve addressing skepticism in the medical community through evidence dissemination. Our review demonstrates that Latin American researchers have started to fill the evidence gap – it is now a matter of translating this growing body of knowledge into practice.

\subsection{Limitations of the Current Evidence and Future Directions}
It is important to acknowledge limitations in the evidence base. Many studies in Latin America are small-scale or preliminary. There is also a bias in publication: countries like Brazil and Mexico, which have more research capacity, generate more reports, whereas countries with fewer resources (perhaps genuinely with cases) might be under-reporting. This means our picture is likely incomplete. We suspect, for example, that some cases of Lyme disease in Central America are simply not documented in the literature – possibly misdiagnosed as something else. Improving diagnostic confirmation (so cases can be published) is key to uncovering the true extent.

Another limitation is that we relied on available literature in selected languages. There may be relevant information in non-indexed sources (e.g., theses, local bulletins) that we did not capture due to our exclusion of grey literature. However, the inclusion of SciELO in our search helps mitigate that to some extent, as SciELO indexes many Latin American journals.

\textbf{Future research priorities} emerging from this review include: (1) \textbf{Isolation and characterization of the BYS agent} – this would be a game-changer for Brazil; (2) \textbf{Seroprevalence studies using improved antigens} in different countries to get a sense of infection rates (for instance, a study using \textit{B. chilensis} antigens in Chile's population around Chiloé Island could confirm if exposure has occurred in humans there); (3) \textbf{Vector competence studies} – e.g., experimentally infecting \textit{Amblyomma cajennense} ticks with \textit{B. burgdorferi} to see if they can transmit and if the spirochete changes form, which could mimic the BYS situation; (4) \textbf{Clinical trials or case series} documenting treatment of Latin American Lyme/BYS in a systematic way, to establish optimal therapy lengths; (5) \textbf{Genomic studies} of all Latin American \textit{Borrelia} isolates to understand virulence factors – are there genes in \textit{B. chilensis} or \textit{Ca. B. paulista} that predict it could cause human disease? Conversely, do BYS strains have gene loss (e.g., missing outer surface proteins that make them harder to detect by the immune system)? Genomics could answer such questions.

In conclusion, Latin America is witnessing the \textbf{emergence and recognition of Lyme disease and related borrelioses} after years of uncertainty. The identification of novel \textit{Borrelia} species is both an exciting scientific development and a call to action for public health authorities in the region. With climate change and globalization, ticks and the diseases they carry are expected to spread further – already there is concern that \textit{Ixodes} ticks might expand their range southward and that migratory birds could drop infected ticks in new areas \mycite{1}. Thus, building capacity now to diagnose and manage Lyme disease in Latin America is prudent.

\section{Conclusions}
\setstretch{1.15}
Lyme disease in Latin America has transitioned from a doubtful proposition to a documented reality. Over the last 15 years, \textbf{peer-reviewed evidence has confirmed indigenous \textit{Borrelia} infections in humans across multiple Latin American countries}, ending the debate over whether Lyme disease "exists" in this region. Mexico has established the presence of \textit{B. burgdorferi} sensu stricto causing typical Lyme disease \mycite{1}, and Brazil has delineated a Lyme-like syndrome (Baggio–Yoshinari Syndrome) that, while unusual in presentation, is linked to tick-transmitted \textit{Borrelia} organisms \mycite{1}. Concurrently, at least four \textbf{novel \textit{Borrelia} species} (e.g., \textit{B. chilensis}, \textit{"Ca. B. paulista"}, \textit{B. ibitipoquensis}, \textit{"Ca. B. mahuryensis"}) have been identified in Latin America's ticks and wildlife, underscoring that the spectrum of \textit{Borrelia} in this region extends beyond the known pathogenic species of the Northern Hemisphere.

Key conclusions from our systematic review include:

\begin{itemize}[leftmargin=*,itemsep=4pt]
    \item \textbf{Latin America harbors diverse \textit{Borrelia} genospecies, some of which are capable of causing human illness.} Confirmed human Lyme borreliosis cases (caused by \textit{B. burgdorferi} s.l.) have been reported in countries such as Mexico, Brazil, Argentina, and likely others \mycite{1}. The clinical presentations range from classic EM and arthritis to chronic relapsing febrile syndromes, reflecting the particular \textit{Borrelia} strain and host context.

    \item \textbf{Novel \textit{Borrelia} species are being discovered in Latin America at an accelerating pace}, thanks to improved molecular surveillance. These include new Lyme-group spirochetes (e.g., \textit{B. chilensis} in Chile \mycite{1}) and entirely distinct lineages (e.g., bat-associated Borrelia in the Amazon \mycite{1}). While most novel species have not yet been implicated in human disease, their presence in human-biting ticks means they are potential emerging pathogens.

    \item \textbf{Regional differences are pronounced:} In \textit{temperate areas} (like northern Mexico), Lyme disease closely resembles that of the USA, whereas in \textit{tropical areas} (like Brazil), the disease can manifest with atypical features (e.g., BYS with relapses and autoimmune phenomena) \mycite{1}. Such differences likely stem from distinct vector species and \textit{Borrelia} strains. Awareness of these variations is crucial for clinicians – a one-size-fits-all definition of Lyme disease is not sufficient in the Latin American context.

    \item \textbf{Public health and surveillance} in the region have not fully caught up with these developments. Under-diagnosis and under-reporting are still issues – many cases are likely misdiagnosed as other diseases or simply not recognized. Strengthening surveillance (including making Lyme disease a reportable condition and investing in diagnostic reference labs) is needed so that incidence and distribution can be mapped. This is especially important as climate and land use changes could be increasing suitable habitats for tick vectors, potentially expanding the range of Lyme risk.

    \item \textbf{Diagnostic improvements are on the horizon}, leveraging the knowledge of local \textit{Borrelia} strains. The integration of genomic data from Latin American isolates into diagnostic test design will be important to improve sensitivity. Clinicians should employ a high index of suspicion and consider sending samples for PCR or sequencing when possible, especially in seronegative cases with strong clinical indications.

    \item \textbf{Treatment of Latin American Lyme disease (and BYS)} should follow evidence-based guidelines, which currently align with international practices. Early antibiotic therapy is effective in most cases \mycite{1}. However, the possibility of persistent symptoms (as seen in some BYS patients) means that follow-up is important, and a multidisciplinary approach (including rheumatology or neurology consultation for chronic cases) may be beneficial. There is no indication that Latin American \textit{Borrelia} are more antibiotic-resistant than their northern counterparts; rather, treatment challenges stem from delayed diagnoses.

    \item \textbf{Emerging trend – One Health approach:} Our review underscores the value of a One Health perspective: human Lyme cases in Latin America did not occur in isolation, but alongside findings in ticks, rodents, birds, and other animals. Collaboration between physicians, veterinarians, entomologists, and ecologists has been and will continue to be key to unraveling the complex epidemiology of \textit{Borrelia} in this region \mycite{1}.
\end{itemize}

In conclusion, Lyme disease in Latin America is an evolving field. What was once considered "imported" has proven to be \textit{indigenous}. Moreover, Latin America has contributed significantly to scientific knowledge by revealing new \textit{Borrelia} species and challenging assumptions about the disease's ecology and clinical spectrum. The \textbf{public health implication} is clear: Latin American healthcare systems must incorporate Lyme disease into differential diagnoses of fever/rash and neurological syndromes, improve diagnostic testing availability, and educate both doctors and the public about tick bite prevention and early symptom recognition. From a research standpoint, continuing to characterize local \textit{Borrelia} strains will inform diagnostics, and investigating the clinical outcomes of patients infected with these strains will guide management protocols.

Lyme disease should no longer be seen as a "disease of the North" – it is a \textbf{global disease that in Latin America takes on unique characteristics}. By acknowledging and studying those unique aspects, we can better protect the health of populations in this region. Early recognition and treatment of Lyme borreliosis in Latin America can prevent the long-term complications associated with this infection, just as in other parts of the world. Finally, the discovery of novel \textit{Borrelia} in Latin America raises a broader point: there may be other yet-undetected pathogenic \textit{Borrelia} species worldwide. Vigilant surveillance and open-minded clinical investigation are essential, in Latin America and beyond, to fully understand and combat the expanding world of tick-borne borrelioses.

\clearpage
\section*{References}
\setstretch{1.0}
\begin{enumerate}
\item Binetruy, F., Garnier, S., Boulanger, N., Talagrand-Reboul, É., Loire, E., Faivre, B., Noël, V., Buysse, M., \& Duron, O. (2020). \textit{A novel Borrelia species, intermediate between Lyme disease and relapsing fever groups, in neotropical passerine-associated ticks}. Scientific Reports, 10(1), 10596. \href{https://doi.org/10.1038/s41598-020-66828-7}{https://doi.org/10.1038/s41598-020-66828-7}

\item Colunga-Salas, P., Sánchez-Montes, S., Volkow, P., Ruíz-Remigio, A., \& Becker, I. (2020). \textit{Lyme disease and relapsing fever in Mexico: An overview of human and wildlife infections}. PLoS ONE, 15(9), e0238496. \href{https://doi.org/10.1371/journal.pone.0238496}{https://doi.org/10.1371/journal.pone.0238496}

\item Ivanova, L. B., Tomova, A., González-Acuña, D., Murúa, R., Moreno, C. X., Hernández, C., Cabello, J., Cabello, C., Daniels, T. J., Godfrey, H. P., \& Cabello, F. C. (2014). \textit{Borrelia chilensis, a new member of the Borrelia burgdorferi sensu lato complex that extends the range of this genospecies in the Southern Hemisphere}. Environmental Microbiology, 16(4), 1069–1080. \href{https://doi.org/10.1111/1462-2920.12310}{https://doi.org/10.1111/1462-2920.12310}

\item Jorge, F. R., Muñoz-Leal, S., de Oliveira, G. M. B., Serpa, M. C. A., Magalhães, M. M. L., de Oliveira, L. M. B., Moura, F. B. P., Teixeira, B. M., \& Labruna, M. B. (2023). \textit{Novel Borrelia genotypes from Brazil indicate a new group of Borrelia spp. associated with South American bats}. Journal of Medical Entomology, 60(1), 222–228. \href{https://doi.org/10.1093/jme/tjac160}{https://doi.org/10.1093/jme/tjac160}

\item Labruna, M. B., Faccini-Martínez, Á. A., Muñoz-Leal, S., Szabó, M. P. J., \& Angerami, R. N. (2024). \textit{Lyme borreliosis in Brazil: A critical review on the Baggio–Yoshinari syndrome (Brazilian Lyme-like disease)}. Clinical Microbiology Reviews, 37(4), e00097-24. \href{https://doi.org/10.1128/CMR.00097-24}{https://doi.org/10.1128/CMR.00097-24}

\item Lucca, V., Nuñez, S., Pucheta, M., Radman, N., Rigonatto, T., Sánchez, G., Del Curto, B., Oliva, D., Mariño, B., López, G., Bonin, S., Trevisan, G., \& Stanchi, N. (2024). \textit{Lyme disease: A review with emphasis on Latin America}. Microorganisms, 12(2), 385. \href{https://doi.org/10.3390/microorganisms12020385}{https://doi.org/10.3390/microorganisms12020385}

\item Mancilla-Agrono, L. Y., Banguero-Micolta, L. F., Ossa-López, P. A., Ramírez-Chaves, H. E., Castaño-Villa, G. J., \& Rivera-Páez, F. A. (2022). \textit{Is Borrelia burgdorferi sensu stricto in South America? First molecular evidence of its presence in Colombia}. Tropical Medicine and Infectious Disease, 7(9), 166. \href{https://doi.org/10.3390/tropicalmed7090166}{https://doi.org/10.3390/tropicalmed7090166}

\item Weck, B. C., Serpa, M. C. A., Labruna, M. B., \& Muñoz-Leal, S. (2022). \textit{A novel genospecies of Borrelia burgdorferi sensu lato associated with cricetid rodents in Brazil}. Microorganisms, 10(2), 204. \href{https://doi.org/10.3390/microorganisms10020204}{https://doi.org/10.3390/microorganisms10020204}

\item Yoshinari, N. H., Bonoldi, V. L. N., Bonin, S., Falkingham, E., \& Trevisan, G. (2022). \textit{The current state of knowledge on Baggio–Yoshinari Syndrome (Brazilian Lyme disease-like Illness): Chronological presentation of historical and scientific events observed over the last 30 years}. Pathogens, 11(8), 889. \href{https://doi.org/10.3390/pathogens11080889}{https://doi.org/10.3390/pathogens11080889}
\end{enumerate}

\end{document}